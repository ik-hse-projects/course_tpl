\documentclass[12pt,a4paper]{article}

% Языки, которые могут быть использованы.
\usepackage[english,russian]{babel}

% FreeSerif — альтернатива TimesNewRoman
\usepackage{fontspec}
\defaultfontfeatures{Ligatures={TeX}} 
\setmainfont[Ligatures={TeX}]{FreeSerif}

% Чтобы сделать буковки ещё красивее.
\usepackage{microtype}

% Чтобы легко менять стиль списков
\usepackage[shortlabels]{enumitem}

\usepackage{csquotes}
% https://qna.habr.com/answer?answer_id=1140867
\bibliographystyle{gost2008}
\usepackage[
	style=gost-numeric, % стиль цитирования и библиографии, см. документацию biblatex-gost
	language=auto,  % использовать язык из babel
	autolang=other, % многоязычная библиография
	parentracker=true,
	backend=biber,
	hyperref=false,
	bibencoding=utf8,
	defernumbers=true,
	sorting=ntvy,  % сортировка: имя, заголовок, том, год
]{biblatex}
\addbibresource{bibliography.bib}
\DeclareSourcemap{
    \maps{
        \map{% если @online, то устанавливаем media=eresource.
            \step[typesource=online, fieldset=media, fieldvalue=eresource]
        }
    }
}

% Отсупы как по госту.
\usepackage[
	includeheadfoot,
	left=20mm,
	right=10mm,
	top=0mm,
	headheight=25mm,
	% Так, чтобы нижняя табличка влезла.
	bottom=6\baselineskip,
	footskip=5\baselineskip,
]{geometry}

\savegeometry{original}
\geometry{bottom=10mm}
\savegeometry{nofooter}
\loadgeometry{original}

%% Некоторые переменные, которые появляются более одного раза

% Название курсовой
\newcommand{\docTitle}{
	Программа для прогнозирования детальных физикохимических свойств нефтей
}

% То же самое название, но в родительном падеже
\newcommand{\docTitleGenitive}{
	Программы для прогнозирования детальных физикохимических свойств нефтей
}

% Год написания курсовой.
\newcommand{\YEAR}{
	\the\year{}
}

% RU — код страны
% 17701729 — код НИУ ВШЭ
% 12.34 — регистрационный номер
% 01 — номер редакции документа
% ТЗ — код вида документа
\newcommand{\docId}{RU.17701729.12.34-01 ТЗ}


% Footer and header
\usepackage{fancyhdr}
\pagestyle{fancy}
\fancyhf{}
\chead{
	%\bf % Сделать жирненьким
	\thepage\\
	\docId
}
\fancyhead[RO]{%
    % В правый верхний колотитул пихаю номер приложения
    {\ifnum\value{addendum}>0 ПРИЛОЖЕНИЕ \theaddendum \fi}
}
\cfoot{%
	\begin{tabular}{| l | l | l | l | l |}
		\hline & & & & \\
		\hline Изм. & Лист & № докум. & Подп. & Дата \\
		\hline \docId &  &  &  &  \\
		\hline Инв. № подл. & Подп. и дата & Взам. инв. № & Инв. № дубл. & Подп. и дата № \\
		\hline
	\end{tabular}
}
\fancypagestyle{nofooter}{%
	\fancyfoot{}%
}

\renewcommand{\headrulewidth}{0pt}
\renewcommand{\footrulewidth}{0pt}

% Картинки, вращение и всё такое.
\usepackage{graphicx}

% Повернуть на 90 градусов
\newcommand{\rot}[2]{\rotatebox[origin=c]{90}{\enspace\parbox{#1 - 0.5em}{#2}}}

% Повторить #1 раз текст #2: \Repeat{#1}{#2}
\usepackage{expl3}
\ExplSyntaxOn
\cs_new_eq:NN \Repeat \prg_replicate:nn
\ExplSyntaxOff

% Продвинутые таблицы
\usepackage{tabularx}
\usepackage{multirow}
\usepackage{xltabular}
% И сразу делаем чтобы колонки X центрировались по вертикали
\def\tabularxcolumn#1{m{#1}}
\newcolumntype{Y}{>{\centering\arraybackslash}X}

% Подсчет числа страниц
\usepackage{lastpage}

% Расчет всяких размеров
\usepackage{calc}

% Названия разделов по центру
\usepackage{titlesec}
\titleformat*{\section}{\centering\Large\bfseries}
% И с точкой в конце.
\titlelabel{\thetitle.\quad}

% После названия раздела надо делать отступы у абзацев.
\usepackage{indentfirst}

% https://tex.stackexchange.com/a/347803
\usepackage{varwidth}

\newcommand{\placename}{
	\underline{\hspace{4cm}}
}
\newcommand{\placedate}{
	«\underline{\hspace{1em}}» \underline{\hspace{3cm}} \YEAR г.
}

\newcounter{addendum}
\newcommand{\addendum}[1]{
    \stepcounter{addendum}
    \section*{#1}
    \addcontentsline{toc}{section}{Приложение \arabic{addendum}: #1}%
}

% TODOшечки
\usepackage{todonotes}

% Добавляем гипертекстовое оглавление в PDF
% hyperref должен быть последним
\usepackage[hidelinks]{hyperref}

\begin{document}
	\selectlanguage{russian}

	\input{titul/main.tex}
	
	% На всякий случай, хотя вообще не очень нужно.
	\loadgeometry{original}
	
	\section*{АННОТАЦИЯ}

Техническое задание – это основной документ, оговаривающий набор требований и
порядок создания программного продукта, в соответствии с которым производится разработка
программы, ее тестирование и приемка.

Настоящее Техническое задание на разработку «Клиент-серверного приложения «Охота на лис»» содержит следующие разделы: «Введение», «Основание для разработки», «Назначение разработки», «Требования к программе», «Требования к программным документам»,
«Технико-экономические показатели», «Стадии и этапы разработки» и «Порядок контроля и приёмки».

В разделе «Введение» указано наименование и краткая характеристика области применения «Клиент-серверного приложения «Охота на лис»».

В разделе «Основания для разработки» указан документ на основании, которого ведётся разработка и наименование темы разработки.

В разделе «Назначение разработки» указано функциональное и эксплуатационное назначение программного продукта.

Раздел «Требования к программе» содержит основные требования к функциональным характеристикам, к надёжности, к условиям эксплуатации, к составу и параметрам технических средств, к информационной и программной совместимости, к маркировке и упаковке, к транспортировке и хранению, а также специальные требования.

Раздел «Требования к программным документам» содержит предварительный состав программной документации и специальные требования к ней.

Раздел «Технико-экономические показатели» содержит ориентировочную экономическую эффективность, предполагаемую годовую потребность, экономические преимущества разработки «Программы поиска маршрута китайского почтальона».

Раздел «Стадии и этапы разработки» содержит стадии разработки, этапы и содержание работ.

В разделе «Порядок контроля и приемки» указаны общие требования к приемке работы.

Настоящий документ разработан в соответствии с требованиями:
\begin{enumerate}[1)]
	\item ГОСТ 19.101-77 Виды программ и программных документов;
	\item ГОСТ 19.102-77 Стадии разработки;
	\item ГОСТ 19.103-77 Обозначения программ и программных документов;
	\item ГОСТ 19.104-78 Основные надписи;
	\item ГОСТ 19.105-78 Общие требования к программным документам;
	\item ГОСТ 19.106-78 Требования к программным документам, выполненным печатным способом;
	\item ГОСТ 19.201-78 Техническое задание. Требования к содержанию и оформлению. Изменения к данному Техническому заданию оформляются согласно ГОСТ 19.603-78, ГОСТ 19.604-78.
\end{enumerate}

\clearpage
	
	\clearpage
		\renewcommand*\contentsname{\centering СОДЕРЖАНИЕ}
		\tableofcontents
	\clearpage
	
	% Основное содержимое
    \section{ГЛОССАРИЙ}
\begin{itemize}
	\item TODO
\end{itemize}

\clearpage     % Глоссарий
	\section{ВВЕДЕНИЕ}

\noindent\subsection{Наименование программы}

Клиент-серверное приложение «Охота на лис» с многопользовательским режимом.

\noindent\subsection{Краткая характеристика области применения}

Программа предназначена для осуществления игры в «охоту на лис», используя мобильные устройства участников, коммуницирующие с сервером. Суть игры заключается в поиске («пеленгации») заранее отмеченных точек («передатчиков») при помощи специального мобильного приложения.

Приложение может применяться как в исключительно развлекательных целях, так и для подготовки людей в области спортивной радиопеленгации.

\clearpage
 % Введение
	\section{ОСНОВАНИЯ ДЛЯ РАЗРАБОТКИ}

\subsection{Документы, на основании которых ведется разработка}
    Разработка программы ведётся на основании учебного плана программы «Программная инженерия»
    

\subsection{Наименование темы разработки}
    Наименование темы разработки – «\docTitle».

\clearpage      % Основания для разработки
	\section{НАЗНАЧЕНИЕ РАЗРАБОТКИ}

\subsection{Функциональное назначение}

Программа должна позволять организаторам игры задавать координаты \emph{передатчиков}, которые после загружаются на сервер, и позволять игрокам видеть направление до выбранного ими \emph{передатчика} относительно их текущего местоположения, запрашивая все необходимые данные с сервера.

\subsection{Эксплуатационное назначение}
Организаторы игры задают координаты точек, а игроки должны их потом найти.

\clearpage        % Назначение разработки
	\section{ТРЕБОВАНИЯ К ПРОГРАММЕ}

\subsection{Требования к функциональным характеристикам}
\label{requirements.features}
    \subsubsection{Требования к составу выполняемых функций}
        \begin{enumerate}[series=requirements]
        	\item TODO
        \end{enumerate}

    \subsubsection{Требования к временным характеристикам}

    \subsubsection{Требования к организации входных данных}
    
    \subsubsection{Требования к организации выходных данных}

\subsection{Требования к интерфейсу}
\label{requirements.interface}
    \begin{enumerate}
    	\item TODO
    \end{enumerate}

\subsection{Требования к надёжности}
\label{requirements.quality}
    \subsubsection{Требования к обеспечению надёжного (устойчивого) функционирования программы}
Чтобы приложение работало надёжно, необходимо соблюдать следующие рекомендации:
\begin{enumerate}
	\item Использовать последнюю доступную версию операционной системы.
	\item Своевременно устанавливать все обновления.
	\item Не допускать заражения устройства компьютерными вирусами, троянами и т.п.
	\item Отключить антивирусное программное обеспечение.
\end{enumerate}

    \subsubsection{Время восстановления после отказа}
Время восстановления после отказа из-за перебоев со связью, электричеством и другими неполадками в оборудовании равняется времени необходимому на устранения проблемы и повторному запуск приложения.

    \subsubsection{Отказы из-за некорректных действий оператора}
Никакие действия оператора не должны приводить к отказу. Для этого требуется ограничить права администратора используя средства ОС.

\subsection{Условия эксплуатации}
Приложение должно эксплуатироваться только в местах, где доступен сигнал GPS.

\subsection{Требования к составу и параметрам технических средств}
Требуется мобильное устройство на платформе Android с версией ОС не менее Android 5.0 с объемом оперативной памяти не менее 3 гигабайт.

\subsubsection{Требования к информационным структурам и методам решения}
    \subsubsection{Требования к информационным структурам и методам решения}

    \subsubsection{Требования к программным средствам, используемым программой}
    
    \subsubsection{Требования к исходным кодам и языкам программирования}

    \subsubsection{Требования к защите информации и программы}

\subsection{Требования к маркировке и упаковке}

\subsection{Требования к транспортировке и хранению}

\subsection{Специальные требования}
    Специальные требования к данной программе не предъявляются.

\clearpage % Требования
	\section{ТРЕБОВАНИЯ К ПРОГРАММНОЙ ДОКУМЕНТАЦИИ}

\subsection{Предварительный состав программной документации}
\label{docs.list}

\begin{enumerate}
    \item «\docTitle». Техническое задание (ГОСТ 19.201-78 \cite{gost:19.201-78});
    \item «\docTitle». Программа и методика испытаний (ГОСТ 19.301-79 \cite{gost:19.301-79});
    \item «\docTitle». Пояснительная записка (ГОСТ 19.404-79);
    \item «\docTitle». Текст программы (ГОСТ 19.401-78).

\end{enumerate}

\subsection{Специальные требования к программной документации}
\label{docs.extra}

\begin{enumerate}
    \item Все документы к программе должны быть выполнены и оформлены в соответствии с
    ГОСТ 19.101-77, ГОСТ 19.103-77, ГОСТ 19.104-78, ГОСТ 19.105-78, с ГОСТ 19.106-78 и ГОСТ к этому виду документа (см. п.\ref{docs.list}).
    \item Внесение изменений в документацию необходимо проводить в соответствии с ГОСТ 19.603-78 \cite{gost:19.603-78}, ГОСТ 19.604-78 \cite{gost:19.604-78}.
\end{enumerate}

\clearpage
         % Требования к программной документации
	\section{ТЕХНИКО-ЭКОНОМИЧЕСКИЕ ПОКАЗАТЕЛИ}

\subsection{Ориентировочная экономическая эффективность}
В рамках данной работы расчет экономической эффективности не предусмотрен.

\subsection{Предполагаемая потребность}
Программа будет очень нужна, правда. Её можно использовать для развлечения, тренировок и ещё всякого разного.

\subsection{Экономические преимущества разработки по сравнению с отечественными и зарубежными образцами или аналогами}
Быстрый поиск в сети Интернет на момент создания приложения не выявил аналогов
данной программы.

\clearpage      % Технико-экономические показатели
	\section{СТАДИИ И ЭТАПЫ РАЗРАБОТКИ}

\subsection{Необходимые стадии разработки, этапы и содержание работ}
% https://docs.cntd.ru/document/1200007628

\begin{xltabular}{\textwidth}{| p{4cm} | p{4cm} | X |}
	\hline
	Стадии & Этапы работ & Содержание работ \\\hline\endhead
	\multirow{15}{*}{Техническое задание}
		& \multirow{4}{4cm}{Основание необходимости разработки программы}
			& Постановка задачи \\\cline{3-3}
		&	& Сбор исходных материалов \\\cline{3-3}
		&	& Выбор и обоснование критериев эффективности и качества разрабатываемой программы \\\cline{3-3}
		&	& Обоснование необходимости проведения научно-исследовательских работ \\\cline{2-3}
		& \multirow{5}{4cm}{Научно-исследовательские работы}
			& Определение структуры входных и выходных данных \\\cline{3-3}
		&	& Предварительный выбор методов решения задач \\\cline{3-3}
		&	& Обоснование целесообразности применения ранее разработанных программ \\\cline{3-3}
		&	& Определение требований к техническим средствам \\\cline{3-3}
		&	& Обоснование принципиальной возможности решения поставленной задачи \\\cline{2-3}
		& \multirow{6}{4cm}{Разработка и утверждение технического задания}
			& Определение требований к программе. \\\cline{3-3}
		&	& Разработка технико-экономического обоснования разработки программы. \\\cline{3-3}
		&	& Определение стадий, этапов и сроков разработки программы и документации на неё. \\\cline{3-3}
		&	& Выбор языков программирования. \\\cline{3-3}
		&	& Определение необходимости проведения научно-исследовательских работ на последующих стадиях. \\\cline{3-3}
		&	& Согласование и утверждение технического задания. \\\cline{1-3}
	\multirow{4}{*}{Рабочий проект}
		& Разработка программы
			& Программирование и отладка программы \\\cline{2-3}
		& Разработка программной документации
			& Разработка программных документов в соответствии с требованиями ГОСТ 19.101-77 \cite{gost:19.101-77} \\\cline{2-3}
		& \multirow{2}{4cm}{Испытания программы}
			& Разработка, согласование и утверждение порядка и методики испытаний \\\cline{3-3}
		&	& Корректировка программы и программной документации по результатам испытаний\\\cline{1-3}
	\multirow{2}{*}{Внедрение}
		& \multirow{2}{4cm}{Подготовка и передача программы}
			& Подготовка и передача программы и программной документации для сопровождения и изготовления \\\cline{3-3}
		&	& Оформление и утверждение акта о передаче программы на сопровождение и изготовление \\\cline{1-3}
\end{xltabular}

\subsection{Cроки разработки}
\begin{enumerate}
	\item TODO
\end{enumerate}

\clearpage       % Стадии и этапы раработки
	\section{ПОРЯДОК КОНТРОЛЯ И ПРИЁМКИ}

\subsection{Виды испытаний}
Проводится контроль функциональных требований, представленных в техническом
задании.
Тестирование делится на несколько этапов:
\begin{enumerate}
	\item Проверка возможности добавлять \emph{передатчики}.
	\item Проверка работы сервера посредством отправки на него GET-запросов для получении
	информации о заданных организаторами игры \emph{передатчиков}.
	\item Проверка работы клиента и корректности вычисления направления до точек.
\end{enumerate}

\subsection{Общие требования к приёмке работы}
Приём программного продукта происходит при полной работоспособности программы при
различных входных данных, при выполнении указанных в разделе \ref{requirements.features} настоящего документа
функций, при выполнении требований указанных в пункте \ref{requirements.quality} настоящего документа и при
наличии полной документации к программе, указанной в пункте \ref{docs.list}, выполненной в соответствии
со специальными требования указанными в пункте \ref{docs.extra} настоящего технического задания.

\clearpage    % Порядок контроля и приёмки

    % Ниже следуют приложения
	\input{content/bibliography.tex} % Ручками редактировать не надо, лучше юзать bibtex (см. bibliography.bib)

    % Дальше снова не приложения
    \setcounter{addendum}{0}

	\input{changes.tex}       % Лист регистрации изменений
\end{document}