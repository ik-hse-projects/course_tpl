\section{ТРЕБОВАНИЯ К ПРОГРАММЕ}

\subsection{Требования к функциональным характеристикам}
\label{requirements.features}

\subsubsection{Требования к составу выполняемых функций}

Организаторы игры могут:
\begin{enumerate}
	\item Задавать координаты точек. \\
	\item Загружать заданные точки на сервер м получать их идентификаторы.
\end{enumerate}

После загрузки точки на сервер, никто больше не может увидеть её координаты. Идентификатор точки должен представлять из себя строку, содержащую только видимые символы ASCII \cite{wiki:ASCII} и быть длиной не более 30 символов.

Игроки могут:
\begin{enumerate}
	\item Получать список координат точек с сервера по списку идентификаторов. \\
	\item Выбирать точки. \\
	\item Видеть направление до выбранных точек.
\end{enumerate}

Важно заметить, что игроки не могут видеть координаты точек.

\subsubsection{Требования к временным характеристикам}
Интерфейс должен реагировать на действия пользователя в течении пяти секунд после нажатия.

\subsection{Требования к интерфейсу}

Интерфейс должен быть интуитивно-понятным.

\subsection{Требования к надёжности}
\label{requirements.quality}

\subsubsection{Требования к обеспечению надёжного (устойчивого) функционирования программы}
Чтобы приложение работало надёжно, необходимо соблюдать следующие рекомендации:
\begin{enumerate}
	\item Использовать последнюю доступную версию операционной системы.
	\item Своевременно устанавливать все обновления.
	\item Не допускать заражения устройства компьютерными вирусами, троянами и т.п.
	\item Отключить антивирусное программное обеспечение.
\end{enumerate}

\subsubsection{Время восстановления после отказа}
Время восстановления после отказа из-за перебоев со связью, электричеством и другими неполадками в оборудовании равняется времени необходимому на устранения проблемы и повторному запуск приложения.

\subsubsection{Отказы из-за некорректных действий оператора}
Никакие действия оператора не должны приводить к отказу. Для этого требуется ограничить права администратора используя средства ОС.

\subsection{Условия эксплуатации}
Приложение должно эксплуатироваться только в местах, где доступен сигнал GPS.

\subsection{Требования к составу и параметрам технических средств}
Требуется мобильное устройство на платформе Android с версией ОС не менее Android 5.0 с объемом оперативной памяти не менее 3 гигабайт.

\subsubsection{Требования к информационным структурам и методам решения}
Требования к методам решения не предъявляются.

\subsubsection{Требования к программным средствам, используемым программой}
Программа должна использовать Xamarin.

\subsubsection{Требования к исходным кодам и языкам программирования}
Программа должна быть написана на языке программирования C\# 8.0, а целевой платформой должна быть указана .NET Core 3.1. В качестве интегрированной среды разработки программы должна быть использована среда Microsoft Visual Studio 2019 или старше или JetBrains Rider 2020.2 или старше.

\subsubsection{Требования к защите информации и программы}
Требования к защите информации и программы не предъявляются.

\subsection{Требования к маркировке и упаковке}
Программа доставляется пользователю через Play Market, соответственно требования к маркировке и упаковке отсутствуют.

\subsection{Требования к транспортировке и хранению}
Приложение, установленное на мобильном устройстве, не нуждается в особых условиях транспортировки и хранения.

\subsubsection{Специальные требования}
Специальные требования к данной программе не предъявляются.

\clearpage